\documentclass[a4paper]{article}

\usepackage{amsmath,amssymb,amsthm}
\newtheorem{proposition}{Proposition}

\renewcommand{\vec}[1]{\boldsymbol{#1}}

\title{Counting Exact Covers}
\author{Guillermo A. P\'erez}


\begin{document}
\maketitle

\section{Satisfiability and Covers}
We are interested in the connections between two decision problems: the
satisfiability problem and the generalized exact cover problem.
We will need definitions for propositional formulas and (set) covers.

\subsection{Propositional logic}
Let $\varphi$ be a propositional formula over Boolean variables $\vec{x} = (x_1,
x_2, \dots, x_n)$. We say $\varphi$ is in conjunctive normal form (CNF) if it is of the
form:
\(
  \bigwedge_{i=1}^m \psi_i(\vec{x})
\)
where the $\psi_i$, called clauses, are disjunctions of possibly negated
variables. We refer to variables occurring in a clause as literals. If they are
negated, then they are negative literals, otherwise they are positive. We
sometimes write $x \in \psi$ to denote that the positive literal $x$
occurs in the clause $\psi$; and $\overline{x} \in \psi$ to denote that the
negative literal $\lnot x$ occurs in it.

\subsection{Set covers}
Let $P$ and $S$ be two disjoint finite sets. We sometimes call $P$ the primary
set, to be covered exactly, and $S$ the secondary set, with a soft cover
constraint. A collection $\mathcal{C}$ of subsets of $P \cup S$ covers
$P$ exactly and $S$ softly if for all elements $p \in P$ there is exactly one set in
$\mathcal{C}$ that contains $p$ and for all elements $s \in S$ there is at
most one set in $\mathcal{C}$ that contains $s$. We call the pair $(P,S)$ a
target pair.

\subsection{Decision problems}
The
\textsc{SAT} problem asks whether a given formula $\varphi$ has a satisfying
valuation. That is, whether there exists some $\vec{a} \in \mathbb{B}^n$ such
that $\varphi(\vec{a}) = 1$.  The \textsc{GXC} problem asks, for given target
pair $(P,S)$, and a collection $\mathcal{A}$ of subsets of $P \cup S$, whether
there is a subset $\mathcal{C}$ of $\mathcal{A}$ that covers $P$ exactly and
$S$ softly.

\begin{proposition}
  There is a logspace many-one reduction from \textsc{SAT} to \textsc{GXC}.
\end{proposition}
To prove the claim above, we can use the following construction. Consider an
instance of \textsc{SAT} given as in the definitions earlier, i.e.
$\varphi(\vec{x}) = \bigwedge_{i=1}^m \psi_i(\vec{x})$ is the CNF formula over
$\vec{x} = (x_1, \dots, x_n)$. Then, we define the primary set to be the set
of all variables union the set of all clauses; and the secondary set to
consist of elements $\ell_{i,j}$ that we will use to encode the occurrence of
variables in clauses. In symbols, we set:
\begin{align*}
  P & {} = \{x_1, x_2, \dots\} \cup \{\psi_i \mid 1 \leq i \leq m\}\\
  S & {} = \{\ell_{i,j} \mid 1 \leq i \leq m, 1 \leq j \leq n\}.
\end{align*}
Now, for the collection of sets, we define $\mathcal{A}$ to be the union of
three collections we describe below.
\begin{enumerate}
  \item The true sets per variable $\bigcup_{j=1}^{n} \left\{\{x_j\} \cup \{
    \ell_{i,j} \mid \overline{x_j} \in \psi_i\} \right\}$
  \item The false sets per variable $\bigcup_{j=1}^{n} \left\{\{x_j\} \cup \{
    \ell_{i,j} \mid x_j \in \psi_i\} \right\}$
  \item The clause-occurrence sets $\bigcup_{i=1}^{m} \{ \{ \psi_i, x_j \} \mid
    1 \leq j \leq n, x_j \in \psi_i \text{ or } \overline{x_j} \in \psi_i\}$
\end{enumerate}

\begin{proposition}
  For any collection $\mathcal{C} \subseteq \mathcal{A}$ that covers $P$
  exactly and $S$ softly we can recover a valuation $\vec{a} \in \mathbb{B}^n$
  such that $\varphi(\vec{a}) = 1$. Conversely, for any valuation $\vec{a} \in
  \mathbb{B}^n$ with $\varphi(\vec{a}) = 1$ we can choose a collection
  $\mathcal{C} \subseteq \mathcal{A}$ that covers $P$ exactly and $S$ softly.
\end{proposition}
\begin{proof}
  Note that, since the variables are in the primary set and the only the true
  and false sets of $x_j$ contain $x_j$, any collection $\mathcal{C} \subseteq
  \mathcal{A}$ that
  covers $P$ exactly corresponds to a valuation $\vec{a} \in \mathbb{B}^n$ of
  the variables. Furthermore, having a clause-occurrence set $\{\psi_i, x_j\}$
  in a collection $\mathcal{C} \subseteq \mathcal{A}$ that covers $P$ exactly means that the
  corresponding valuation $\vec{a} \in \mathbb{B}^n$ satisfies $\psi_i$, i.e.
  $\psi_i(\vec{a}) = 1$. To see this, note that the true set of $x_j$ would make
  an exact cover of $P$ with that pair impossible if $x_j$ occurs as a negative
  literal in $\psi_i$. Similarly, the false set of $x_j$ precludes an exact
  cover of $P$ with that pair if $x_j$ occurs as a positive literal in it.  Now,
  since the clauses are in the primary set, this means any collection covering
  $P$ exactly corresponds to a valuation that satisfies all clauses. Conversely,
  a satisfying valuation $\vec{a} \in \mathbb{B}^n$ of $\varphi$ can be used to
  construct a collection $\mathcal{C} \subseteq \mathcal{A}$ that covers $P$
  exactly and $S$ softly. However, while the choice of true and false sets is given direclty by
  the valuation, there can be multiple options for the choice of
  clause-occurrence sets. For instance, one can choose the first clause $\psi_i$
  (i.e. the clause with minimal index $i$) satisfied by the valuation $\vec{a}$.
\end{proof}


\section{Counting Problems}

\end{document}
